\documentclass[a4paper, 12pt]{article}
\usepackage{extsizes}
\usepackage{amsmath,amsthm,amssymb}
\usepackage[pdftex,bookmarks=true]{hyperref} % добавляет в pdf файл ссылки. Важно: должен стоять перед mathtext!
\usepackage{mathtext}
\usepackage{textcomp}
\usepackage[T1,T2A]{fontenc}
\usepackage{float}
\usepackage{mathtools}
\usepackage{indentfirst}
%\usepackage{physics}
\usepackage{wrapfig}
\usepackage[dvips, pdftex]{graphicx}
\usepackage{caption}
\graphicspath{{pictures/}}
\DeclareGraphicsExtensions{.png, .jpg}
\usepackage[utf8]{inputenc}
\usepackage[english,russian]{babel}
\usepackage{multirow}
\captionsetup[figure]{labelfont={bf},labelformat={default},labelsep=period,name={Рис.}}

% ВАЖНО: ПЕРЕД ПУБЛИКАЦИЕЙ ПРОВЕРИТЬ НА ОСТАВШИЕСЯ МЕТКИ TODO !

\DeclareMathOperator*\lowlim{\underline{lim}}
\DeclareMathOperator*\uplim{\overline{lim}}

\usepackage[left=20mm, top=20mm, right=20mm, bottom=20mm, footskip=10mm]{geometry}
\usepackage{wrapfig}

\usepackage{comment}

\graphicspath{ {images/} }
\usepackage{multicol}
\setlength{\columnsep}{1cm}

\addto\captionsrussian{\def\refname{Список используемой литературы}}

\begin{document}
\begin{titlepage}
  \begin{center}
    \large
    Московский физико-технический институт \\ (национальный исследовательский университет)
     
    \vspace{0.75cm}
     
    \textbf{Физтех-школа электроники, фотоники и молекулярной физики}
     
    \vfill
    
    \Large
    \textbf{Лабораторная работа} \\
    \textit{Лазерный гироскоп} \\
     
    \vfill
\end{center}

\vbox{
\hfill
\vbox{
\hbox{\textbf{Выполнили:}}
\hbox{Яшин Прохор Б04-006}
\hbox{Белостоцкий Артемий Б04-006}
\hbox{Мазуренко Илья Б04-006}
\hbox{Шарапов Алексей Б04-006}
\hbox{Ионидис Екатерина Б04-006}
\hbox{Наталья Плюскова Б04-004}
%
}%
} 


\vfill
 
\begin{center}
    г. Долгопрудный \\
\end{center}
\end{titlepage}

\newpage

\newpage

\input{0_Цель_работы}

\input{1_Теория}

\input{2_Эксперимент}

\input{3_Вывод}

\newpage

\input{4_Задачи}

\end{document}

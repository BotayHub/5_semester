\documentclass[12pt]{article}
\usepackage{amsmath}
\usepackage{mathtext}
\usepackage[T2A]{fontenc}
\usepackage[utf8]{inputenc}
\usepackage[russian]{babel}
\usepackage[left=2cm, right=2cm, top=2.00cm]{geometry}
\usepackage[section]{placeins}

\title{Отчёт о выполнении лабораторной работы 4.2 <<Исследование энергетического
  спектра $\beta$-частиц и определение их максимальной энергии при помощи
  магнитного спектрометра>>}
\author{Маланчук С.В.}
% \author{Плохой Красный Коммунист}
\date{01.10.2020}

\usepackage{natbib}
\usepackage{graphicx}

\begin{document}

\begin{flushright}
    Выполнила:
    \\
    Маланчук С.В.,
    \\
    878 группа
    % \it{Плохой Красный Коммунист}
\end{flushright}

\begin{center}
    \begin{Large}
        \textbf{Отчёт о выполнении лабораторной работы 4.2 <<Исследование энергетического
  спектра $\beta$-частиц и определение их максимальной энергии при помощи
  магнитного спектрометра>>}
    \end{Large}
\end{center}

% \maketitle

\parindent=1cm \textbf{Цель работы:} с помощью магнитного спектрометра
исследовать энергетический спектр $\beta$-частиц при распаде ядер $^{137}Cs$ и
определить их максимальную энергию.

\parindent=1cm \textbf{Оборудование:} магнитный $\beta$-спектрометр с <<короткой
линзой>>, радиоактивный источник $^{137}Cs$, форвакуумный насос, вакуумметр, счетчик,
высоковольтный и низковольтный выпрямители, ЭВМ.
\begin{center}
    \textbf{Теория}
\end{center}

\textbf{$\beta$-распад}~--- самопроизвольное превращения ядер, при котором
массовое число остается постоянным, а заряд изменяется на единицу.
\\
\textbf{Электронный распад:}
\begin{equation*}
    ^A_ZX \to ^A_{Z+1}X+e^-+\tilde{\nu}
\end{equation*}
где $\tilde{\nu}$~--- антинейтрино.
\begin{equation}
    \label{eq:(1)}
    E_e-E-ck = 0
\end{equation}
где $E_e$~--- максимальная эергия электрона, для которой верно
\begin{equation}
    \label{eq:(2)}
    E = c\sqrt{p^2+m^2c^2}-mc^2
\end{equation}
а $ck$~--- энергия антинейтрино с импульсом $k$. Учтем ($1$) введением в
выражение для $dw$~--- вероятности того, что при распаде электрон вылетит с
импульсом $d^3\vec{p}$, а антинейтрино~--- с импульсом $d^3\vec{k}$~--- $\delta$-функции
\begin{equation}
    \label{eq:(3)}
    \delta(E_e-E-ck)
\end{equation}
(не равна нулю только при выполнении ($1$)).
\\
Тогда
\begin{equation}
    \label{eq:(4)}
    dw = D\delta(E_e-E-ck)d^3\vec{p}d^3\vec{k} = D\delta(E_e-E-ck)p^2dpk^2dkd\Omega_ed\Omega_{\tilde{\nu}}
\end{equation}
где $D$~--- коэффициент пропорциональности, $d\Omega_e$ и
$d\Omega_{\tilde{\nu}}$~--- элементы телесных углов направлений вылета электрона
и нейтрино.
\\
Кроме того,
\begin{equation}
    \label{eq:(5)}
    dN = N_0dw
\end{equation}
где $N_0$~--- общее число распадов, а $dN$~--- число таких распадов, где
импульсы электрона и антинейтрино от $\vec{p}$ до $\vec{p}+d\vec{p}$ и от
$\vec{k}$ до $\vec{k}+d\vec{k}$ соответственно.
\\
Интегрируя,
\begin{equation}
    \label{eq:(6)}
    dN = \frac{16\pi^2N_0}{c^2}Dp^2(E_e-E)^2dp
\end{equation}
Переходя к $dE$ от $dp$, получим
\begin{equation}
    \label{eq:(7)}
    dE = \frac{c^2p}{E+mc^2}dp
\end{equation}
откуда величина, выражающая форму $\beta$-спектра
\begin{equation}
    \label{eq:(8)}
    N(E) = \frac{dN}{dE} = N_0Bcp(E+mc^2)(E_e-E)^2 = N_0B\sqrt{E(E+2mc^2)}(E_e-E)^2(E+mc)^2
\end{equation}
где $B = \displaystyle \frac{16\pi^2}{c^4}D$.
\\
Переходя к нерелятивистскому приближению,
\begin{equation}
    \label{eq:(9)}
    \frac{dN}{dE}\approx \sqrt{E}(E_e-E)^2
\end{equation}
\textbf{Конверсионные электроны}~--- такие электроны, которые излучаются в
процессе отдачи энергии возбужденными ядрами атомов.
\\
У приборов:
\begin{itemize}
    \item зависимость фокусного растояния <<линзы>> от тока через катушку:
    \begin{equation}
        \label{eq:(10)}
        \frac{1}{f} \propto \frac{I^2}{p^2_e}
    \end{equation}
    \item импульс сфокусированных электронов
    \begin{equation}
        \label{eq:(11)}
        p_e = kI
    \end{equation}
    где $k$~--- константа прибора.
\end{itemize}
    \begin{equation}
        \label{eq:(12)}
        N(p_e) \simeq W(p_e)\Delta p_e
    \end{equation}
    где $\Delta p_e$~--- разрешающая способность. Дифференцируя ($10$),
    \begin{equation}
        \label{eq:(13)}
        \Delta p_e = \frac{1}{2}\frac{\Delta f}{f}p_e
    \end{equation}
    Итак, окончательно,
    \begin{equation}
        \label{eq:(14)}
        N(p_e) = CW(p_e)p_e
    \end{equation}
    где $C= const$.
\newpage
\begin{center}
    \textbf{Ход работы}
\end{center}

\begin{center}
    \textbf{Измерения и наблюдения}
\end{center}
Время одного измерения~--- $100$ с.
\\
Фон до измерения: $2,3494 \pm 0,153$ частиц/с.
\\
\begin{center}
    \begin{tabular}{|c|c|c|}
      \hline
 $I$, А & $int$, частиц/с & Примечание \\ \hline
0,00 & 2,579 &  \\ \hline
0,20 & 2,169 &  \\ \hline
0,40 & 2,599 &  \\ \hline
0,60 & 2,169 &  \\ \hline
0,80 & 2,879 &  \\ \hline
1,00 & 4,929 &  \\ \hline
1,20 & 7,298 &  \\ \hline
1,40 & 8,777 &  \\ \hline
1,60 & 9,547 &  \\ \hline
1,80 & 8,797 &  \\ \hline
2,00 & 8,198 &  \\ \hline
2,20 & 6,758 &  \\ \hline
2,40 & 5,558 &  \\ \hline
2,60 & 4,319 &  \\ \hline
2,80 & 3,479 &  \\ \hline
3,00 & 3,959 &  \\ \hline
3,10 & 5,248 &  \\ \hline
3,15 & 5,948 &  \\ \hline
3,20 & 6,078 &  \\ \hline
3,20 & 5,828 & повторное \\ \hline
3,25 & 4,329 &  \\ \hline
3,25 & 4,149 & повторное \\ \hline
3,30 & 2,609 &  \\ \hline
3,30 & 2,849 & повторное \\ \hline
3,40 & 1,660 &  \\ \hline
3,40 & 1,849 & повторное \\ \hline
3,50 & 1,380 &  \\ \hline
3,50 & 1,680 & повторное \\ \hline
3,50 & 1,660 & повторное \\ \hline
3,60 & 1,460 &  \\ \hline
3,60 & 1,560 & повторное \\ \hline
3,80 & 3,149 &  \\ \hline
4,00 & 2,729 &  \\ \hline
4,20 & 1,809 &  \\ \hline
   \end{tabular}
\end{center}
Фон после измерения: $2,4694 \pm 0,153$ частиц/с.
\begin{center}
    \textbf{Обработка}
\end{center}
Интенсивность фона в среднем: $2,4094 \pm 0,306$ частиц/с.
\\
Обработав, получаем:
\begin{center}
    \begin{tabular}{|c|c|c|c|c|c|c|c|c|c|c|c|}
      \hline
      $I$, А & $\sigma_I$ & $n$, 1/с & $\sigma_{n}$ & $n-n_{b}$, 1/c & $\sigma_{n-n_{b}}$ & $p$, кэВ/с & $\sigma_{p}$ & $T$, кэВ & $\sigma_{T}$ & mkFm & $\sigma_{\text{mkFm}}$ \\ \hline
      0,00 & 0,005 & 2,579 & 0,161 & 0,1696 & 0,3136 & 0,0 & 50,8 & 0,0 & 0,0 & 0,0 & 0,0 \\ \hline
0,20 & 0,005 & 2,169 & 0,147 & -0,2404 & 0,3003 & 63,3 & 46,6 & 3,9 & 5,7 & 0,0 & 0,0 \\ \hline
0,40 & 0,005 & 2,599 & 0,161 & 0,1896 & 0,3142 & 126,6 & 51,0 & 15,4 & 12,4 & 305,7 & 438,1 \\ \hline
0,60 & 0,005 & 2,169 & 0,147 & -0,2404 & 0,3003 & 189,9 & 46,6 & 34,1 & 16,7 & 0,0 & 0,0 \\ \hline
0,80 & 0,005 & 2,879 & 0,170 & 0,4696 & 0,3227 & 253,2 & 53,7 & 59,3 & 25,2 & 170,1 & 112,5 \\ \hline
1,00 & 0,005 & 4,929 & 0,222 & 2,5196 & 0,3750 & 316,5 & 70,3 & 90,0 & 40,0 & 281,9 & 114,9 \\ \hline
1,20 & 0,005 & 7,298 & 0,270 & 4,8886 & 0,4231 & 379,8 & 85,5 & 125,6 & 56,6 & 298,7 & 113,8 \\ \hline
1,40 & 0,005 & 8,777 & 0,296 & 6,3676 & 0,4493 & 443,1 & 93,8 & 165,3 & 70,0 & 270,5 & 95,4 \\ \hline
1,60 & 0,005 & 9,547 & 0,309 & 7,1376 & 0,4620 & 506,4 & 97,8 & 208,3 & 80,5 & 234,4 & 75,5 \\ \hline
1,80 & 0,005 & 8,797 & 0,297 & 6,3876 & 0,4496 & 569,7 & 93,9 & 254,2 & 83,8 & 185,9 & 52,5 \\ \hline
2,00 & 0,005 & 8,198 & 0,286 & 5,7886 & 0,4393 & 633,0 & 90,6 & 302,4 & 86,6 & 151,1 & 38,2 \\ \hline
2,20 & 0,005 & 6,758 & 0,260 & 4,3486 & 0,4130 & 696,3 & 82,3 & 352,6 & 83,3 & 113,5 & 25,5 \\ \hline
2,40 & 0,005 & 5,558 & 0,236 & 3,1486 & 0,3888 & 759,6 & 74,6 & 404,4 & 79,4 & 84,8 & 17,7 \\ \hline
2,60 & 0,005 & 4,319 & 0,208 & 1,9096 & 0,3608 & 822,9 & 65,8 & 457,5 & 73,1 & 58,5 & 12,5 \\ \hline
2,80 & 0,005 & 3,479 & 0,187 & 1,0696 & 0,3395 & 886,2 & 59,0 & 511,8 & 68,2 & 39,2 & 10,1 \\ \hline
3,00 & 0,005 & 3,959 & 0,199 & 1,5496 & 0,3520 & 949,5 & 63,0 & 567,1 & 75,2 & 42,5 & 9,1 \\ \hline
3,10 & 0,005 & 5,248 & 0,229 & 2,8386 & 0,3821 & 981,2 & 72,5 & 595,1 & 88,0 & 54,8 & 9,8 \\ \hline
3,15 & 0,005 & 5,948 & 0,244 & 3,5386 & 0,3969 & 997,0 & 77,2 & 609,1 & 94,3 & 59,8 & 10,3 \\ \hline
3,20 & 0,005 & 6,078 & 0,247 & 3,6686 & 0,3995 & 1012,8 & 78,0 & 623,2 & 96,0 & 59,4 & 10,1 \\ \hline
3,25 & 0,005 & 4,329 & 0,208 & 1,9196 & 0,3611 & 1028,6 & 65,9 & 637,4 & 81,6 & 42,0 & 8,0 \\ \hline
3,30 & 0,005 & 2,609 & 0,162 & 0,1996 & 0,3145 & 1044,5 & 51,1 & 651,6 & 63,8 & 13,2 & 11,4 \\ \hline
3,50 & 0,005 & 1,380 & 0,117 & -1,0294 & 0,2705 & 1107,8 & 37,2 & 708,7 & 47,6 & 0,0 & 0,0 \\ \hline
3,60 & 0,005 & 1,460 & 0,121 & -0,9494 & 0,2738 & 1139,4 & 38,2 & 737,5 & 49,5 & 0,0 & 0,0 \\ \hline
3,80 & 0,005 & 3,149 & 0,177 & 0,7396 & 0,3305 & 1202,7 & 56,2 & 795,5 & 74,3 & 20,6 & 6,1 \\ \hline
4,00 & 0,005 & 2,729 & 0,165 & 0,3196 & 0,3182 & 1266,0 & 52,3 & 854,0 & 70,5 & 12,6 & 7,0 \\ \hline
4,20 & 0,005 & 1,809 & 0,134 & -0,6004 & 0,2875 & 1329,3 & 42,6 & 912,9 & 58,5 & 0,0 & 0,0 \\ \hline
   \end{tabular}
\end{center}
Получаем графики:
\begin{figure}[ht]
    \centering
    \includegraphics[scale=0.3]{plot1.png}
    \label{fig:plot1}
\end{figure}
\begin{figure}[ht]
    \centering
    \includegraphics[scale=0.3]{plot2.png}
    \label{fig:plot2}
\end{figure}
\begin{figure}[ht]
    \centering
    \includegraphics[scale=0.3]{plot3.png}
    \label{fig:plot3}
\end{figure}
\begin{figure}[ht]
    \centering
    \includegraphics[scale=0.3]{plot4.png}
    \label{fig:plot4}
\end{figure}
\begin{figure}[ht]
    \centering
    \includegraphics[scale=0.3]{plot5.png}
    \label{fig:plot5}
\end{figure}
\begin{figure}[ht]
    \centering
    \includegraphics[scale=0.3]{plot6.png}
    \label{fig:plot6}
\end{figure}
\FloatBarrier
\newpage
\begin{center}
    \textbf{Обсуждение}
\end{center}
Выполнив данную лабораторную работу, мы измерили максимальную энергию
$\beta$-спектра, она оказалась равной около $540 \pm 50$ кэВ. Рассмотрели зависимости
интенсивности от импульса, кинетической энергии электронов и силы тока в катушке
спектрометра. Построили график Ферми-Кюри.
\begin{center}
    \textbf{Вывод}
\end{center}
Максимальная энергия $\beta$-спектра равна примерно $540 \pm 50$ кэВ, что
соответствует известному значению.
\end{document}

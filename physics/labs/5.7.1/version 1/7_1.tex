\documentclass[a4paper,12pt]{article}
\usepackage[utf8]{inputenc}
\usepackage[T2A]{fontenc}
\usepackage[english, russian]{babel}
\usepackage{amsthm}
\usepackage{amsmath}
\usepackage{amssymb}
\usepackage{tikz}
\usepackage{textcomp}
\usepackage{esint}
\usepackage[unicode]{hyperref}
\usepackage{indentfirst}
\usepackage{algorithm}
\usepackage[noend]{algpseudocode}
\usepackage{amsmath,amsfonts,amssymb,amsthm,mathtools}
\usetikzlibrary{positioning,arrows}
\usepackage{graphicx}
\setlength{\topmargin}{-0.5in}
\setlength{\textheight}{9.1in}
\setlength{\oddsidemargin}{-0.4in}
\setlength{\evensidemargin}{-0.4in}
\setlength{\textwidth}{7in}
\setlength{\parindent}{0ex}
\setlength{\parskip}{1ex}
\usepackage{siunitx}

\usepackage{multicol}
\usetikzlibrary{trees}
\usepackage{fancyhdr}
\usepackage{gensymb}

\newcommand{\bbR}{\mathbb R}
\newcommand{\eps}{\varepsilon}
\newcommand{\bbN}{\mathbb N}
\newcommand{\dif}{\mathrm{d}}

\pagestyle{fancy}
\makeatletter
\fancyhead[L]{}

\fancyfoot[R]{\thepage}
\fancyfoot[C]{}

\renewcommand{\maketitle}{
	\noindent{\bfseries\scshape\large\@title\ \mdseries\upshape}\par
	\noindent {\large\itshape\@author}
	\vskip 2ex}
\makeatother




\begin{document}
	
\Large \textbf { \begin{center}
		Работа 7.1\\ Измерение углового распределения жесткой компоненты космического излучения \\
		Селюгин Михаил, 876 \\
\end{center}}


	\section{Теория вопроса}
	
	Космические лучи --- это стабильные частицы и ядра атомов, зародившиеся и ускоренные до больших энергий вне Земли, изотропно падающие на границу земной атмосферы (первичное космическое излучение), а также различные частицы, рожденные ими при взаимодействии с ядрами атомов воздуха (вторичное космическое излучение).
	
	Первичное космическое излучение --- протоны($90\%$) и $\alpha$ частицы ($10\%$). Для протонов сечение взаимодействия с ядрами атомов, содержащихся в воздухе, близко к геометрическому и равняется 	
	$$\sigma_{p \ air} \simeq \pi R^2 = \pi (R_0 A^{\frac1{3}}) = 300\text{мб}$$
	
	Тогда слой атмосферы при таких сечениях есть $8\div 12$ свободных пробегов протонов.
	
	Вторичные частицы включают в себя пионы, каоны и гипероны. Для пиона время жизни $\tau_0 = 2,5\cdot 10^{-8}$ и схема распада
	$$\pi^{\pm} \rightarrow \mu^{\pm} + \nu$$
	
	Но также возможно взаимодействие с ядрами атомов воздуха. Какое из двух событий произойдет зависит от плотности атмосферы. Пробег для распада:	
	$$L_{\text{расп}} = \beta c \tau_0 \gamma = \frac{c\tau_0 E_\pi}{m_\pi c^2}$$
		
		
	А для взаимодействия:
	$$L_{\text{вз}} = \frac1{\sigma_{\pi\ air} n} = \frac{A} {\sigma_{\pi\ air} \rho N}, \ \sigma_{\pi\ air}\simeq \frac2{3} \sigma_{p \ air} \simeq 200 \text{мб} $$
	
	Тогда вероятность взаимодействия равна вероятности распада при
	$$ E_\pi =  \frac{A m_\pi c^2} {\sigma_{\pi\ air} \rho N c \tau_0}$$
	
	Мюоны же преимущественно вызывают ионизацию воздуха. Для прохождения всей атмосферы энергия должна составлять порядка $2$ГэВ. Распад мюона происходит со временем жизни $\tau_0 = 2\cdot 10^{-6}$ по каналам:
	$$\mu^{+} \rightarrow e^+ + \widetilde{\nu_e} + \nu_\mu, \ \mu^{-} \rightarrow e^- + \widetilde{\nu_\mu} + \nu_e$$
	
	Время жизни увеличивается $\tau = \tau_0 \gamma$. При этом распадный пробег мюона с энергией $2$ГэВ есть 
	$$L_{\text{расп}} = \beta c \tau_0 \gamma \approx 12km$$
	
	Также в сильных взаимодействиях рождаются нейтральные пионы, распадающиеся на два гамма-кванта. Их время жизни мало и они не успевают взаимодействовать с атомными ядрами. Гамма кванты же в поле атомных ядер рождают электрон-позитронные пары.
	
	Процесс продолжается лавинообразно и один нейтральный пион может дать начало лавине, число частиц в которой достигает $10^5$.  Как следтсвие одна частица с энергией $10^{20}$эВ дает на уровне моря лавину с числом частиц порядка $10^{10}.$
	
	Исследования показывают, что интенсивность распределения космических лучей резко зависит от направления, увеличиваясь при переходе к вертикальному. При этом по вертикали мюоны проходят порядка $L_0 \simeq 15km$. Тогда 
	$$\Delta L = L_0(\frac1{\cos \theta} - 1)$$
	$$P(\Delta L) = 1 - \exp(-\Delta L / L_\text{расп}), L_\text{расп} = c\tau$$
	
	А также число дошедших мюонов уменьшается за счет поглощения в веществе по закону
	$$P_1(\theta) \propto (\cos\theta)^{1,6}$$
	
	А из-за распада мюонов
	$$P_2(\theta) = \exp(-L(\theta)/L)$$
	
	Тогда отношение числа мюонов, идущих под зенитным углом $\theta$, к числу вертикально падающих есть
	$$\frac{N(\theta)}{N(0)} =  \frac{P_1(\theta)P_2(\theta)}{P_1(0)P_2(0)} = (\cos\theta)^{1,6} \frac{e^{-L(\theta)/L}}{e^{-L(0)/L}}$$
	
	Учитывая $L(\theta) = L_0 / \cos\theta$, получаем оценку на время жизни $\tau_0$.
		
	
	\section{Экспериментальная установка}
	
	
	
	Установка регистрирует те частицы, которые летят внутри обозначенного телесного угла. Схема регистрирует истинные и случайные совпадения.
	
	Число случайных совпадений $N = 2\tau N_1 N_2$, где $N_1, N_2$ --- число импульсов за единицу времени от каждого счетчика.
	
	Величину разрешающего времени стараются максимально уменьшить. Для данной геометрии $\tau \simeq (1\div 2) \cdot 10^{-6}$c.
	
	\begin{figure}[h!]
		\centering
		\includegraphics[width=\linewidth]{kosmos.png}
	\end{figure}

	\section{Результаты измерений}
	
	\begin{enumerate}
		
		\item Измерение угловой зависимости интенсивности жесткой компоненты.
		\item Оценка времени жизни мюона.
	\end{enumerate}
	
	
	\newpage
	\section{Обработка результатов}
	
	{\bf 1. } Для обработки результатов введем следующие обозначения:
	\begin{itemize}
		\item $\theta$ --- угол наклона телескопа;
		\item $N$ --- показания счетчика;
		\item $I_\text{ф}$ --- частота фонового излучения;
		\item  $I_\text{п}$ --- частота двойных совпадений;
		\item $I$ --- частота двойных совпадений с учетом фона.
	\end{itemize}

	Внесем данные в таблицу.
	
	\includegraphics[width=1.1\linewidth]{7_1_table.png}
	
	{\bf 2. Оценка фона} 
	
	Проведем оценку фонового излучения исходя из показателей счетчика при угле наклона телескопа $\theta = 90^\circ$ и получим
	$$I_\text{ф} = 0,035\pm 0,002\ (\eps = 5\%)$$
	
	\newpage{\bf 3.} 
	По таблице $1$ построим график
	
	\includegraphics[width=1.1\linewidth]{7_1_graph.png}
	
	Теоретически была выведена зависимость 
	$$I = I_0 \cos^n\theta$$
	
	В логарифмическом масштабе угол наклона как раз равняется показателю степени $n$. 
	C помощью МНК был определен угол наклона линии тренда
	$$n = 1,71\pm 0,07\ (\eps = 4\%)$$
	
	{\bf 4. Оценка числа случайных совпадений}
	Число случайных совпадений определяется формулой
	$$I_{rand} = 2\tau I_1 I_2 = 2\tau \frac{N_1N_2}{T^2}$$
	Для проведенного эксперимента
	$\tau = 10^{-7}c$, $T = 300c$, $N_1 = 3175$, $N_2 = 2961$.
	
	Отсюда, $I_{rand} \approx 2\cdot 10^{-5}$, что на несколько порядков меньше величины измеряемых в эксперименте двойных сопадений.
	
	{\bf 5. Оценка времени жизни мюона}
	
	Из теории время жизни мюона:
	$$\tau_0 = \dfrac{L_0 (\cos\theta - 1)}{\cos\theta (\ln\frac{N(\theta)}{N(0)} - 1,6\ln\cos \theta)} \cdot \dfrac {m_\mu c^2}	{\beta c E_\mu}$$
	
	Проведем оценку для $45^\circ$, учитывая табличные значения полной энергии, энергии покоя и величины свободного пробега мюона. 
	$$\tau_0 = 6,2 \pm 0,6 \text{мкс}\ (\eps=8\% )$$
	
	{\bf 6. Оценка погрешностей}
	\begin{enumerate}
		\item Для построения графика.
		
		$\delta \theta = 1^\circ$ - погрешность измерения угла наклона.
		Тогда систематическая погрешность $\delta(\ln\cos\theta) = \tan \theta\cdot \delta\theta \leq 0.05$. 
		Исходя из этого, возьмем погрешность по оси абсцисс равной $0,05$
		
		Случайную погрешность измерения фона находим из того, что счетчик при $\theta = 90^\circ$ зафиксировал $N = 21$ двойное совпадение  
		$$\eps_{ I_\text{ф}} = \frac1{21} \approx 0,05 $$
		
		Тогда возьмем $\eps_I = 5\%$ и рассчитаем погрешность по оси ординат как
		$$\eps _{\frac{I}{I_0}} = \sqrt{\eps_{I}+ \eps_{I_0} } = \sqrt{2\eps_{I}} \approx 7\% \Rightarrow \delta\frac{I}{I_0} \leq 0,07$$
		$$\delta(\ln\frac{I}{I_0}) = \delta\frac{I}{I_0} \frac{I_0}{I}\leq 0,2$$
		
		\item Для оценки времени жизни мюона.
		
		Воспользуемся приведенными выше выкладками и рассчитаем погрешности $\delta(\ln\frac{I}{I_0}), \ \delta(\ln\cos\theta)$ в точке $\theta = 60^\circ$.
		
		$\delta(\ln\frac{I}{I_0}) = 0,06$
		
		$\delta(\ln\cos\theta) = 0,03$
		
		Тогда для их суммы $\delta_+ = 0,11, \ \eps_+ = 5\%$ и для итоговой погрешности получаем	
		$$\eps_\tau = \sqrt{2\eps^2_{\cos} + \eps^2_{+}} \approx 10\% $$
	\end{enumerate}
	\section{Вывод}
	В данной работе было изучено угловое распределение компоненты космического излучения. Теоретически была предсказана зависимость $I = I_0 \cos^n\theta$, а в ходе эксперимента был также получен показатель $n = 1,71\pm0,07$. Причем точки с хорошей точностью легли на прямую, хотя не производилось отсеивания мягкой компоненты космического излучения. Из этого можно сделать вывод, что в условиях эксперимента вклад мягкой компоненты незначителен. Также была проведена оценка вклада случайных совпадений и было получено, что он на порядок меньше величины фиксируемых двойных совпадений.
	
	На последнем этапе работы было оценено время жизни мюона: $\tau_0 = 6,2\pm 0,6$мкс. Результат по порядку величины совпал с табличным значением.
	
	
\end{document}
	
	
	
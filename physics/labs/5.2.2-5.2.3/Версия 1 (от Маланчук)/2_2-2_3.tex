\documentclass[12pt]{article}
\usepackage{amsmath}
\usepackage{mathtext}
\usepackage[T2A]{fontenc}
\usepackage[utf8]{inputenc}
\usepackage[russian]{babel}
\usepackage[left=2cm, right=2cm, top=2.00cm]{geometry}
\usepackage[section]{placeins}

\title{Отчёт о выполнении лабораторных работ 2.2 и 2.3 <<Изучение спектров
  атомов водорода и дейтерия>> и <<Изучение молекулярного спектра йода>>}
\author{Маланчук С.В.}
% \author{Плохой Красный Коммунист}
\date{15.10.2020}

\usepackage{natbib}
\usepackage{graphicx}

\begin{document}

\begin{flushright}
    Выполнила:
    \\
    Маланчук С.В.,
    \\
    878 группа
    % \it{Плохой Красный Коммунист}
\end{flushright}

\begin{center}
    \begin{Large}
        \textbf{Отчёт о выполнении лабораторных работ 2.2 и 2.3 <<Изучение спектров
          атомов водорода и дейтерия>> и <<Изучение молекулярного спектра йода>>}
    \end{Large}
\end{center}

% \maketitle

\parindent=1cm \textbf{Цель работы:} исследовать спектральные закономерности в
оптических спектрах водорода и дейтерия, вычислить постоянные Ридберга,
потенциалы ионизации и изотопические сдвиги линий для этих изотопов водорода;
исследовать спектр поглощения паров йода в видимой области, вычислить энергию
колебательного кванта молекулы йода и энергию ее диссоциации в основном и
возбужденном состояниях.

\parindent=1cm \textbf{Оборудование:} призменный монохроматор, неоновая,
ртутная, водородная лампы, кювета с парами йода, лампа накаливания.

\begin{center}
    \textbf{Теория}
\end{center}
Связь длины волны, заряда и линии спектра для водородоподобных атомов:
\begin{equation}
    \label{eq:(1)}
    \frac{1}{\lambda_{mn}} = RZ^2\left( \frac{1}{n^2} - \frac{1}{m^2} \right)
\end{equation}
где $R$~--- постоянная Ридберга.
\begin{equation}
    \label{eq:(2)}
    E = E_{\text{эл}} + E_{\text{кол}} + E_{\text{вращ}}
\end{equation}
\begin{equation}
    \label{eq:(3)}
    \psi = \psi_{\text{эл}}\psi_{\text{кол}}\psi_{\text{вращ}}
\end{equation}
Энергия колебательного кванта возбужденного состояния молекулы йода:
\begin{equation}
    \label{eq:(4)}
    h\nu_2 = \frac{h\nu_{1,5}-h\nu_{1,0}}{5}
\end{equation}
\newpage
\begin{center}
    \textbf{Ход работы}
\end{center}

\begin{center}
    \textbf{Измерения и наблюдения}
\end{center}
Калибровочные измерения.
\begin{center}
    \begin{tabular}{cc}
      Неон & Ртуть \\
      \begin{tabular}{|c|c|r|}
        \hline
        Угол барабана, $^{\circ}$ & $\lambda$, \AA &Полоса \\ \hline
        2008&5852&22 \\ \hline
        1947&5401&23 \\ \hline
        1912&5341&24 \\ \hline
        2260&5945&20 \\ \hline
        2314&6074&17 \\ \hline
        2326&6096&16 \\ \hline
        2344&6143&15 \\ \hline
        2354&6164&14 \\ \hline
        2396&6267&12 \\ \hline
        2422&6334&10 \\ \hline
        2448&6402&8 \\ \hline
        2486&6507&7 \\ \hline
        2544&6678&4 \\ \hline
        2622&6929&2 \\ \hline
      \end{tabular}      &     \begin{tabular}{|c|c|r|}
                                 \hline
                                 Угол барабана, $^{\circ}$ & $\lambda$, \AA &Полоса \\ \hline
                                 2612&6907&К1 \\ \hline
                                 2380&6234&К2 \\ \hline
                                 2175&5791&1 \\ \hline
                                 2164&5770&2 \\ \hline
                                 1988&5461&3 \\ \hline
                                 1578&4916&4 \\ \hline
                                 908&4358&5 \\ \hline
                                 366&4047&6 \\ \hline
                               \end{tabular}
    \end{tabular}
\end{center} 
Измерения для веществ:
\begin{center}
    \begin{tabular}{cc}
      Водород & Йод \\
      \begin{tabular}{|c|r|}
        \hline
        Угол барабана, $^{\circ}$ &Полоса \\ \hline
        2499&$H_\alpha$ \\ \hline
        1514&$H_\beta$ \\ \hline
        879&$H_\gamma$ \\ \hline
        464&$H_\delta$ \\ \hline
      \end{tabular}      &     \begin{tabular}{|c|r|}
                                 \hline
                                 Угол барабана, $^{\circ}$ & Полоса \\ \hline
                                 2338&$h\nu_{1,0}$ \\ \hline
                                 2243&$h\nu_{1,5}$ \\ \hline
                                 1840&$h\nu_{\text{гр}}$? \\ \hline
                                 1718&$h\nu_{\text{гр}}$? \\ \hline
                               \end{tabular}
    \end{tabular}
\end{center} 
\begin{center}
    \textbf{Обработка}
\end{center}
По известным длинам волн неона и ртути построим градуировочную кривую.
\\
Погрешность измерения угла поворота барабана $\approx 1^\circ$.
\\
По известной градуировочной кривой определим длины волн спектров испускания
водорода и поглощения йода.
\begin{center}
    \begin{tabular}{cc}
      Водород & Йод \\
      \begin{tabular}{|c|r|}
        \hline
        $\lambda$, \AA &Полоса \\ \hline
        6550&$H_\alpha$ \\ \hline
        4860&$H_\beta$ \\ \hline
        4340&$H_\gamma$ \\ \hline
        4095&$H_\delta$ \\ \hline
      \end{tabular}      &     \begin{tabular}{|c|r|}
                                 \hline
                                 $\lambda$, \AA & Полоса \\ \hline
                                 6140 &$h\nu_{1,0}$ \\ \hline
                                 5920 &$h\nu_{1,5}$ \\ \hline
                                 5245 &$h\nu_{\text{гр}}$? \\ \hline
                                 5090 &$h\nu_{\text{гр}}$? \\ \hline
                               \end{tabular}
    \end{tabular}
\end{center}
\begin{center}
     \includegraphics[scale=0.4]{plot1.png}
 \end{center}
\begin{center}
     \includegraphics[scale=0.4]{plot2.png}
 \end{center}
\begin{center}
     \includegraphics[scale=0.4]{plot3.png}
 \end{center}
Примем погрешность определения длины волны за $\pm 5$\AA.
\\
В лабораторной работе исследуется серия Бальмера для водорода; для нее $n = 2$,
а $m = 3,4,5,6$ для $H_\alpha, H_\beta, H_\gamma, H_\delta$ соответственно
(уравнение (\ref{eq:(1)})).
\\
Найдем, соответственно, значения $R$ для различных линий, чтобы, усреднив,
получить истинное значение.
\begin{center}
    \begin{tabular}{|c|c|c|c|}
        \hline
        $\lambda$, \AA &Полоса & $R$ & $\sigma_R$\\ \hline
        6550&$H_\alpha$ & $1,099 \cdot 10^{-3}$ \AA$^{-1}$ &$8,4 \cdot 10^{-7}$\AA$^{-1}$\\ \hline
        4860&$H_\beta$ & $1,097 \cdot 10^{-3}$ \AA$^{-1}$ &$1,13 \cdot 10^{-6}$\AA$^{-1}$ \\ \hline
        4340&$H_\gamma$ & $1,097 \cdot 10^{-3}$ \AA$^{-1}$ &$1,26 \cdot 10^{-6}$\AA$^{-1}$ \\ \hline
        4095&$H_\delta$ & $1,099 \cdot 10^{-3}$ \AA$^{-1}$ &$1,34 \cdot 10^{-6}$\AA$^{-1}$ \\ \hline
    \end{tabular}
\end{center}
Итак, $R= 1,098 \cdot 10^{-3} \pm 5,77 \cdot 10^{-7}$ \AA$^{-1} = 1,098 \cdot
10^5 \pm 5,77\cdot 10$ см$^{-1}$.
\\
Вычислим энергии для йода:
\begin{center}
    \begin{tabular}{|c|c|c|c|}
      \hline
      $\lambda$, \AA & Полоса & $E$ & $\sigma_E$\\ \hline
      6140 &$h\nu_{1,0}$ & $2,019$ эВ & $1,6 \cdot 10^{-3}$ эВ\\ \hline
      5920 &$h\nu_{1,5}$ & $2,094$ эВ & $1,8 \cdot 10^{-3}$ эВ\\ \hline
      5245 &$h\nu_{\text{гр}}$? & $2,364$ эВ & $2,3 \cdot 10^{-3}$ эВ\\ \hline
      5090 &$h\nu_{\text{гр}}$? & $2,436$ эВ & $2,4 \cdot 10^{-3}$ эВ\\ \hline
    \end{tabular}
\end{center}
Тогда:
\begin{itemize}
    \item по \ref{eq:(4)} $h\nu_2 = 0,015 \pm 2,4 \cdot 10^{-3}$ эВ~---
    энергия колебательного кванта возбужденного состояния молекулы йода.
    \item $h\nu_{\text{эл}} = h\nu_{1,5} - \displaystyle \frac{h\nu_1}{2} =
    2,081 \pm 1,8 \cdot 10^{-3}$ эВ.
    \item $D_1 = h\nu_{\text{гр}} - E = 1,496 \pm 2,3\cdot 10^{-3}$ эВ~---
    энергия диссоциации молекулы в основном состоянии.
    \item $D_2 = h\nu_{\text{гр}}-h\nu_{\text{эл}} + 5,5h\nu_2 = 0,4375 \pm 3,5
    \cdot 10^{-3}$ эВ~--- энергия диссоциации молекулы в возбужденном состоянии.

\end{itemize}
\begin{center}
    \textbf{Обсуждение}
\end{center}
Выполнив данную лабораторную работу, мы научились строить и использовать
градуировочную кривую монохроматора, изучили спектры испускания атомарного
водорода и поглощения паров молекулярного йода, вычислили постоянную Ридберга
для водородоподобных атомов и энергии диссоциации молекул йода.
\\
Все значения оказались близки к табличным.
\\
Граница схождения спектра поглощения йода, обнаруженная моим напарником,
оказалась ближе к истинной, чем обнаруженная мной; но истинная граница, если
судить по вычисленным энергиям диссоциации, лежит дальше в коротковолновом
диапазоне.
\begin{center}
    \textbf{Вывод}
\end{center}
$R= 1,098 \cdot 10^5 \pm 5,77\cdot 10$ см$^{-1}$, что близко к табличному значению.
\end{document}

\documentclass[12pt]{article}
\usepackage{amsmath}
\usepackage{mathtext}
\usepackage[T2A]{fontenc}
\usepackage[utf8]{inputenc}
\usepackage[russian]{babel}
\usepackage[left=2cm, right=2cm, top=2.00cm]{geometry}
\usepackage[section]{placeins}

\title{Отчёт о выполнении лабораторной работы 10.1 <<Электронный парамагнитный резонанс>>}
\author{Маланчук С.В.}
% \author{Плохой Красный Коммунист}
\date{17.09.2020}

\usepackage{natbib}
\usepackage{graphicx}

\begin{document}

\begin{flushright}
    Выполнила:
    \\
    Маланчук С.В.,
    \\
    878 группа
    % \it{Плохой Красный Коммунист}
\end{flushright}

\begin{center}
    \begin{Large}
        \textbf{Отчёт о выполнении лабораторной работы 10.1 <<Электронный парамагнитный резонанс>>}
    \end{Large}
\end{center}

% \maketitle

\parindent=1cm \textbf{Цель работы:} исследовать электронный парамагнитный
резонанс в молекуле ДПФГ, определить $g$-фактор электрона, измерить ширину линии
ЭПР.

\parindent=1cm \textbf{Оборудование:} радиоспектроскоп, вольтметр, трансформатор
ЛАТР, осциллограф, блок питания, генератор мегагерцового диапазона, образец ДФПГ.
\begin{center}
    \textbf{Теория}
\end{center}

Энергетический уровень электрона в присутствии магнитного поля $B$ расщепляется
на два подуровня. Расстояние между ними равно
\begin{equation}
    \label{eq:(1)}
    \Delta E = E_2 - E_1 = 2 \mu B
\end{equation}
Между уровнями возможны переходы. Они могут возбуждаться внешним высокочастотным
магнитным полем. Резонансное значение частоты определяется из соотношения
\begin{equation}
    \label{eq:(2)}
    \hbar \omega_0 = \Delta E = 2 \mu B
\end{equation}
При переходе с нижнего на верхний уровень квант энергии поглощается, а при
обратном переходе излучается квант той же частоты. Возбуждение электронных
резонансных переходов ЭМ полем с частотой $\omega_0$ называется электронным
парамагнитным резонансом.
\\
Без внешнего высокочастотного поля заселенность верхнего и нижнего уровней $N_u$
и $N_d$ определяется температурой и описывается формулой Больцмана
\begin{equation}
    \label{eq:(3)}
    \frac{N_u}{N_d} = \exp \left( - \frac{\Delta E}{k T} \right)
\end{equation}
Гиромагнитное соотношение
\begin{equation}
    \label{eq:(4)}
    \mu = \gamma \vec{M}
\end{equation}
Если магнитный момент выражается в магнетонах Бора, а механический в единицах
$\hbar$, то связь выражается через фактор Ланде
\begin{equation}
    \label{eq:(5)}
    \frac{\mu}{\mu_B} = \frac{g \vec{M}}{\hbar}
\end{equation}
Можно выразить $g$-фактор через определяемые экспериментально величины
\begin{equation}
    \label{eq:(6)}
    g = \frac{\hbar \omega_0}{\mu_B B}
\end{equation}
\newpage
\begin{center}
    \textbf{Ход работы}
\end{center}

\begin{center}
    \textbf{Измерения и наблюдения}
\end{center}
Резонансная и половинные частоты:
\\
$f_{res}= 126,76 \pm 0,02\text{МГц}$
\\
$f_h= 126,97 \pm 0,02\text{МГц}$
\\
$f_l= 126,49 \pm 0,02 \text{МГц}$
Тогда добротность контура:
$$Q= \frac{f}{\Delta f} = \frac{f_{res}}{f_h - f_l} = \frac{126,76}{0,48} \approx
264,1$$
$$\varepsilon_Q = \varepsilon_{f_{res}} + \varepsilon_{\Delta f} \approx
\varepsilon_{\Delta f} = 0,08$$
$$\sigma_Q = 21$$
$$Q = (2,6 \pm 0,21) \cdot 10^2$$
Пробная катушка:
\\
$N = 45$
\\
$d = 15,2 \pm 0,1 \text{мм}$
\\
Зная также ЭДС $\varepsilon_i = (2,51 \pm 0,01) \cdot 10^{-3}\text{В}$ и частоту $\nu = 50
\text{Гц}$, найдем величину модулирующего поля:
$$B=\sqrt{2} \frac{2 \varepsilon_i}{\pi^2d^2N\nu} \approx 1,52 \cdot 10^{-3}
\text{Тл} = 1,52 \text{мТл}$$
$$\varepsilon_B = \varepsilon_{\varepsilon_i} + 2\varepsilon_d = 0,003 + 0,013
= 0,016$$
$$\sigma_B = 0,024 \text{мТл}$$
$$B = 1,52 \pm 0,024 \text{мТл}$$
Для полуширины на полувысоте линии резонансного поглощения получим
$$\Delta B= \frac{A_{half}}{A_{full}} B = 0,146 \text{мТл}$$
Калибровочный график (напряжение на пробной катушке от внешнего):
\begin{figure}[ht]
    \centering
    \includegraphics[scale=0.3]{plot1.png}
    \label{fig:plot1}
\end{figure}
$$B_0=\frac{V_t}{NS 2 \pi \nu} = 7,03 \pm 0,45 \text{мТл}$$
Тогда
$$g = 1,9 \pm 0,2$$
График зависимости резонансной частоты от силы тока:
\begin{figure}[ht!]
    \centering
    \includegraphics[scale=0.3]{plot2.png}
    \label{fig:plot2}
\end{figure}
% \begin{figure}[ht!]
%     \centering
%     \includegraphics[scale=0.4]{plot1.png}
%     \caption{$\alpha = 0,033$}
%     \label{fig:plot1}
% \end{figure}
\newpage
\begin{center}
    \textbf{Обсуждение}
\end{center}
Выполнив данную лабораторную работу, мы измерили $g$-фактор электрона,
пронаблюдав явление ЭПР. Он оказался равен $g = 1,9 \pm 0,2$, что совпадает с
табличным значением.
\begin{center}
    \textbf{Вывод}
\end{center}
$g$-фактор электрона $g = 1,9 \pm 0,2$, что совпадает с табличным значением.
\end{document}

\documentclass[12pt]{article}
\usepackage{amsmath}
\usepackage{mathtext}
\usepackage[T2A]{fontenc}
\usepackage[utf8]{inputenc}
\usepackage[russian]{babel}
\usepackage[left=2cm, right=2cm, top=2.00cm]{geometry}
\usepackage[section]{placeins}

\title{Отчёт о выполнении лабораторной работы 7.4 <<Исследование поглощения
  вторичного космического излучения в веществе>>}
\author{Маланчук С.В.}
% \author{Плохой Красный Коммунист}
\date{29.10.2020}

\usepackage{natbib}
\usepackage{graphicx}

\begin{document}

\begin{flushright}
    Выполнила:
    \\
    Маланчук С.В.,
    \\
    878 группа
    % \it{Плохой Красный Коммунист}
\end{flushright}

\begin{center}
    \begin{Large}
        \textbf{Отчёт о выполнении лабораторной работы 7.4 <<Исследование поглощения
  вторичного космического излучения в веществе>>}
    \end{Large}
\end{center}

% \maketitle

\parindent=1cm \textbf{Цель работы:} с помощью телескопа, состоящего из двух
сцинтилляционных детекторов, работающих в схеме совпадений, измеряется
зависимость интенсивности космического излучения в лаборатории (на уровне моря)
от толщины поглотителей из различных материалов. На основе этих измерений
определяются эффективные длины поглощения мягкой и жесткой компонент космики в
свинце и железе, абсолютные величины их вертикальных интенсивностей  и сечение
рождения электронно-позитронных пар в этих материалах.

\parindent=1cm \textbf{Оборудование:} телескоп, состоящий из двух
сцинтилляционных детекторов; схема совпадений; свинцовые блоки.
\begin{center}
    \textbf{Теория}
\end{center}
Жесткая компонента:
\begin{equation}
    \label{eq:(1)}
    \pi^+ \to \mu^+ + \nu_\mu, \  \pi^- \to \mu^- + \tilde{\nu}_\mu,
\end{equation}
Электронно-фотонная компонента:
\begin{equation}
    \label{eq:(2)}
    \pi^0 \to \gamma + \gamma
\end{equation}
Сечение образования электронно-позитронных пар:
\begin{equation}
    \label{eq:(3)}
    \sigma_{pair} = \frac{28}{9}Z^2\alpha \left(\frac{e^2}{mc^2}\right)^2\left(\ln\frac{2\hbar \omega}{mc^2} - \frac{109}{42} - \left(\alpha Z\right)^2\right)
\end{equation}
Для свинца
\begin{equation}
    \label{eq:(3)}
    \sigma_{pair} =11Z^2\alpha r_0^2 = 11Z^2\frac{e^6}{\hbar m^2 c^5}
\end{equation}
\newpage
\begin{center}
    \textbf{Ход работы}
\end{center}
\begin{center}
    \textbf{Измерения и наблюдения}
\end{center}
\begin{center}
    \begin{tabular}{|c|c|c|c|}
      \hline
      \# опыта & $t$, с & \# импульсов & Пластины \\ \hline
      1 & 600 & 172 & нет \\ \hline
      2 & 600 & 139 & 1 \\ \hline
      3 & 600 & 147 & 2 \\ \hline
      4 & 600 & 121 & 3 \\ \hline
      5 & 600 & 131 & 4 \\ \hline
      6 & 600 & 115 & 5 \\ \hline
      7 & 600 & 128 & 6 \\ \hline
      8 & 600 & 112 & 7 \\ \hline
      9 & 600 & 120 & 8 \\ \hline
      10 & 600 & 132 & 9 \\ \hline
      11 & 600 & 149 & 9 \\ \hline
   \end{tabular}
\end{center}
\begin{center}
    \begin{tabular}{|c|c|}
      \hline
      \# пластины & $d$, мм \\ \hline
      1 & 19,3 \\ \hline
      2 & 19,1 \\ \hline
      3 & 19,4 \\ \hline
      4 & 19,4 \\ \hline
      5 & 19,3 \\ \hline
      6 & 19,5 \\ \hline
      7 & 20,2 \\ \hline
      8 & 19,2 \\ \hline
      9 & 19,4 \\ \hline
   \end{tabular}
\end{center}
\begin{center}
    \textbf{Обработка}
\end{center}
\begin{center}
    \begin{tabular}{|c|c|c|c|c|c|c|c|}
      \hline
      Опыт & $N_{imp}$ & $\sigma_{N_{imp}}$ & Пластины & $d_{sum},$, мм & $\sigma_{d_{sum}}$, мм & $\displaystyle \frac{N_{imp}(d_{sum})}{N_{imp}(d_{all})} - 1$ & $\sigma$ \\ \hline
1 & 172 & 13,1 & 0 & 0,0 & 0,00 & 0,30 & 0,05  \\ \hline
2 & 139 & 11,8 & 1 & 19,3 & 0,05 & 0,05 & 0,01 \\ \hline
3 & 147 & 12,1 & 2 & 38,4 & 0,10 & 0,11 & 0,02  \\ \hline
4 & 121 & 11,0 & 3 & 57,8 & 0,15 & -0,08 & 0,01  \\ \hline
5 & 131 & 11,4 & 4 & 77,2 & 0,20 & -0,01 & 0,00  \\ \hline
6 & 115 & 10,7 & 5 & 96,5 & 0,25 & -0,13 & 0,02  \\ \hline
7 & 128 & 11,3 & 6 & 116,0 & 0,30 & -0,03 & 0,01  \\ \hline
8 & 112 & 10,6 & 7 & 136,2 & 0,35 & -0,15 & 0,03  \\ \hline
9 & 120 & 11,0 & 8 & 155,4 & 0,40 & -0,09 & 0,02  \\ \hline
10 & 132 & 11,5 & 9 & 174,8 & 0,45 & 0,00 & 0  \\ \hline
    \end{tabular}
\end{center}
Получаем графики:
\begin{figure}[ht]
    \centering
    \includegraphics[scale=0.3]{plot2.png}
    \label{fig:plot2}
\end{figure}
\begin{figure}[ht]
    \centering
    \includegraphics[scale=0.3]{plot1.png}
    \label{fig:plot1}
\end{figure}
\FloatBarrier
Обе зависимости аппроксимируются экспонентами.
\\
У первого графика (число частиц от толщины) она имеет вид $f(x) = 48,8\cdot
\exp(-0,035x) + 121,8$; тогда эффективная длина пробега будет приближенно равна
$28,6$ мм.
\newpage
\begin{center}
    \textbf{Обсуждение}
\end{center}
Выполнив данную лабораторную работу, мы установили зависимость интенсивности
вторичного космического излучения от толщины свинцовых поглотителей. Заметили,
что как интенсивность излучения, так и отношение мягкой компоненты к жесткой
убывает экспоненциально. Нашли эффективную длину пробега (около $28,6$ мм).
\begin{center}
    \textbf{Вывод}
\end{center}
Эффективная длина пробега мягкой компоненты в свинце равна около $28,6$ мм.
\end{document}
